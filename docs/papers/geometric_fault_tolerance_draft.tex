# Polygonal Validation Study  
  
## Hypothesis  
Higher-order polygonal configurations in FPT provide superior fault tolerance   
through geometric symmetry, reducing consensus overhead and maintaining   
coherence under Byzantine failures.  
  
## Methodology  
- **Nodes**: 50 (distributed consensus simulation)  
- **Trials**: 10,000+ per configuration  
- **Disruption Levels**: 10%, 30%, 50% Byzantine failures  
- **Polygons Tested**: Pentagon (5), Heptagon (7), Decagon (10), Hendecagon (11)  
- **Metrics**: Coherence (σ), binding energy (Δ), recovery time  
- **Statistical Analysis**: ANOVA + Tukey HSD post-hoc tests  
  
## Results  
  
### Coherence Under Disruption  
| Disruption | Pentagon | Heptagon | Decagon | Hendecagon |  
|------------|----------|----------|---------|------------|  
| 10%        | 0.890    | 0.918    | 0.932   | 0.937      |  
| 30%        | 0.782    | 0.870    | 0.912   | 0.918      |  
| 50%        | 0.623    | 0.785    | 0.874   | 0.880      |  
  
### Key Findings  
1. **Low Disruption (10%)**: Marginal gains (5.2% improvement)  
2. **Medium Disruption (30%)**: Moderate scaling (17.4% improvement)  
3. **High Disruption (50%)**: Significant advantage (41.3% improvement)  
4. **Statistical Significance**: F = 45.2, p < 1e-6  
  
### Phase Transition  
Golden ratio threshold (φ ≈ 0.618) represents critical coherence boundary:  
- σ < 0.618: "Normal state" (resistive, prone to collapse)  
- σ > 0.618: "Supercoherent state" (lossless propagation)  
  
## Interpretation  
Polygonal symmetry reduces consensus rounds through geometric precomputation,  
with diminishing returns above ~11 sides. Heptagon (7) offers optimal   
balance of efficiency and fault tolerance for most use cases.  
  
## References  
- Simulation code: `polygon_validation.py`  
- Data: `data/polygonal_results.csv`  
- Statistical analysis: `analysis/anova_results.txt`